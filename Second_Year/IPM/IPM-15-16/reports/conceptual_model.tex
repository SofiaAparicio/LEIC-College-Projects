\documentclass{article}
\usepackage[utf8]{inputenc}
\usepackage[top=1.25in, bottom=1.25in, left=1.5in, right=1.5in]{geometry}

\begin{document}
\section*{Modelo Conceptual}



\subsection*{Metáfora}
 Tendo em conta o propósito da mesa BarISTa pode ser estabelecida uma relação metafórica entre esta e as \textit{vending machines} que estamos habituados a ver no dia a dia, em especial com os \textit{cashiers} táteis que se encontram em algumas cadeias grandes como a McDonalds. Ambos têm como intuito facilitar o processo de compra, bem como o atendimento, melhorando, assim, a experiência de utilização, usando mecanismos simples (como ecrãs táteis com símbolos intuitivos) que o utilizador já conhece de outros aparelhos similares.

\subsection*{Conceitos}
\subsubsection*{Objetos e Atributos}
\begin{itemize}
\item\textbf{bebida} (atributos: nome, conjunto de ingredientes, preço)
\item\textbf{música} (atributos: nome,intérprete,género,álbum)
\item\textbf{ingrediente} (atributos: nome, quantidade)
\item\textbf{jogo} (atributos: nome,jogadores,score board)
\item\textbf{pedido} (atributos: número do pedido, bebidas, preço total)
\item\textbf{lista de bebidas} (atributos: bebidas disponíveis)
\item\textbf{lista de ingredientes} (atributos: ingredientes disponíveis)
\item\textbf{catalogo de musicas} (atributos: musicas)
\end{itemize}

\subsubsection*{Operações}
Escolher uma bebida;\\
Personalizar uma bebida;\\
Finalizar uma bebida;\\
Votar numa música;\\
Selecionar ingrediente;\\
Escolher a quantidade do ingrediente;\\
Iniciar jogo;\\
Pausar jogo;\\
Terminar jogo;\\
Convidar jogadores;\\
Efetuar pedido;\\
Submeter pedido;\\
Pesquisar numa lista;\\
Ordenar itens;\\

\subsection*{Relações entre conceitos}
Uma bebida possuí vários ingredientes;\\
Um pedido tem uma ou mais bebidas;\\
Um jogo pode exigir o uso de bebidas para jogar;\\
A lista de bebidas possuem varias bebidas que se encontram disponíveis no bar;
Cada lista de ingredientes possui os ingredientes possíveis de adicionar a uma bebida;
O Catalogo de Musicas tem as musicas que e possível votar para tocar



\subsection*{Mapeamento}
Ingredientes no barISTa \textless -\textgreater O nível de açúcar numa maquina de vendas de cafe\\
Bebida no barISTa \textless -\textgreater   Uma bebida numa maquina de vendas\\
Pedido no barISTa \textless -\textgreater  A escolha da bebida numa maquina de vendas
Lista de bebidas no barISTa \textless -\textgreater O conjunto de bebidas disponiveis numa maquina de venda\\


\section*{Cenários de Atividades}

\subsection*{Cocktail personalizado}
\textbf{Cenário}: Após um dia de trabalho,o João decide ir a um bar beber uma copo. Quando chega ao bar, o João resolve  \underline{escolher} um Long Island. Vai \underline{pesquisar na lista das bebidas} e apercebesse que esta não tem nenhum Long Island. Consequentemente, o João decide \underline{personalizar} a sua \underline{bebida}. A partir da\underline{lista de ingredientes} disponibilizados pela funcionalidade Cocktail personalizado, o João \underline{seleciona}, tequila, gin, whiskey, rum, vodka, Coca-Cola e sumo de laranja (\underline{ingredientes} de um Long Island). Para cada \underline{ingrediente}, ele \underline{escolhe a quantidade} que deseja que esteja presente. Quando já não deseja adicionar mais nenhum \underline{ingrediente}, o João \underline{finaliza} a sua \underline{bebida personalizada}. No entanto, o João deseja pedir outra \underline{bebida} para o seu amigo José que tinha acabado de chegar ao bar. O José quer uma caipirinha,\underline{bebida} esta que consta na \underline{lista}, só que em vez de limão quer morango. Então, o João decide \underline{personalizar} a caipirinha. Assim, através da funcionalidade Cocktail personalizado, o João \underline{seleciona} a caipirinha e altera o limão pelo morango(\underline{ingrediente} disponibilizado pela funcionalidade), \underline{finalizando}, de seguida, o cocktail. Dado que nenhum dos dois deseja pedir mais nada, o João \underline{submete o pedido}. Quando o pedido de ambos chega, eles ficam bastante agradados pois as suas bebidas estão exatamente como desejavam. No final cada um deles pode mandar vir uma conta separada com as suas \underline{bebidas}\\

\subsection*{Votação para Música Ambiente}
\textbf{Cenário}: Dado que é sexta-feira à noite, o João decide ir a um bar para desanuviar de uma semana cansativa. Ao chegar ao bar, senta-se e \underline{efetua} o seu \underline{pedido}. Enquanto espera pelo seu \underline{pedido} vai apreciando a  \underline{música} ambiente até que decide ir votar numa \underline{música}. O João como é grande fã de Pink Floyd decide \underline{pesquisar} pela canção "Wish you where here" dos Pink Floyd. Como a canção é uma das \underline{músicas} disponibilizadas na \underline{lista de músicas} o João, após a pesquisa, consegue votar na dita \underline{ música}. Passado alguns minutos, ele decide ir ver se a \underline{música} em que tinha \underline{votado} anteriormente, era uma das cinco mais votadas, de modo a ver se esta poderia ser das próximas a ser produzida. Vê que esta está em primeiro e passados uns minutos esta começa a dar no bar, ficando assim o João bastante contente.

\subsection*{Coletânea de jogos de bebida}
\textbf{Cenário}: Após um dia de estudo intensivo o João, o José e o Pedro, decidem ir a um bar para se divertirem um pouco, ao chegarem ao bar apercebem-se que lhes apetece jogar um \underline{jogo de bebidas}. Primeiramente,escolhem as suas \underline{bebidas}. Chegadas as \underline{bebidas}, \underline{iniciam} uma partida com 3 \underline{jogadores}. 
Passado algum tempo os 3 amigos,tendo ficado contentes com o \underline{jogo}, decidem jogar uma segunda vez. No entanto decidem jogar com outros dois rapazes que tinham conhecido no bar, que estavam noutra mesa. Assim, \underline{convidam} os dois rapazes para participarem no \underline{jogo} com eles. Ao aperceberem-se das horas o João, o José e o Pedro decidem \underline{terminar o jogo}, dando assim por acabada a noite.





\end{document}